\documentclass{pracamgr}

\usepackage{polski}
\usepackage[utf8]{inputenc}

\author{Maciej Pazurkiewicz}
\nralbumu{248267}

\title{Modelowanie sygnalizacji świetlnej}

\tytulang{Modelling of a traffic lights system}
\kierunek{Informatyka}

\opiekun{dra hab. Sławomira Lasoty\\
  Instytut Informatyki
  }
% miesiąc i~rok:
\date{Wrzesień 2011}

%Podać dziedzinę wg klasyfikacji Socrates-Erasmus:
\dziedzina{ 
11.3 Informatyka\\ 
}

%Klasyfikacja tematyczna wedlug AMS (matematyka) lub ACM (informatyka)
\klasyfikacja{D. Software\\
  D.2. Software engineering\\
  D.2.4. Software/Program verification}

% Słowa kluczowe:
\keywords{}

% Moje makra
\newtheorem{defi}{Definicja}[section]

\begin{document}
\maketitle

\begin{abstract}
streszczenie
\end{abstract}

\tableofcontents

\chapter*{Wprowadzenie}
\addcontentsline{toc}{chapter}{Wprowadzenie}

\chapter{Sygnalizacja świetlna}


\begin{thebibliography}{99}
\addcontentsline{toc}{chapter}{Bibliografia}

\bibitem[Bea65]{beaman} Juliusz Beaman, \textit{Morbidity of the Jolly
    function}, Mathematica Absurdica, 117 (1965) 338--9.

\bibitem[Blar16]{eb1} Elizjusz Blarbarucki, \textit{O pewnych
    aspektach pewnych aspektów}, Astrolog Polski, Zeszyt 16, Warszawa
  1916.
\end{thebibliography}

\end{document}

%%% Local Variables:
%%% mode: latex
%%% TeX-master: t
%%% coding: utf-8
%%% End:
