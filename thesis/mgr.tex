\documentclass{pracamgr}

\usepackage{polski}
\usepackage[utf8]{inputenc}

\author{Maciej Pazurkiewicz}
\nralbumu{248267}

\title{Modelowanie sygnalizacji świetlnej}

\tytulang{Modelling of a traffic lights system}
\kierunek{Informatyka}

\opiekun{dra hab. Sławomira Lasoty\\
  Instytut Informatyki
  }
% miesiąc i~rok:
\date{Wrzesień 2011}

%Podać dziedzinę wg klasyfikacji Socrates-Erasmus:
\dziedzina{ 
11.3 Informatyka\\ 
}

%Klasyfikacja tematyczna wedlug AMS (matematyka) lub ACM (informatyka)
\klasyfikacja{D. Software\\
  D.2. Software engineering\\
  D.2.4. Software/Program verification}

% Słowa kluczowe:
\keywords{}

% Moje makra
\newtheorem{defi}{Definicja}[section]
\newcommand{\ang}[1]{(ang.~\emph{#1})}

\begin{document}
\maketitle

\begin{abstract}
streszczenie
\end{abstract}

\tableofcontents

\chapter*{Wprowadzenie}
\addcontentsline{toc}{chapter}{Wprowadzenie}

\chapter{Sygnalizacja świetlna}
\label{c:sygnalizacja}

Niniejszy rozdział zawiera opis systemów sygnalizacji świetlnej, które
będą modelowane blabla. Jest on oparty na raportach przygotowanych na
zlecenie amerykańskiej agencji \emph{Federal Highway Administration}
\cite{timing} \cite{handbook}.

\section{Przegląd współczesnych systemów sygnalizacji}
\label{s:przeglad}

Sygnalizacją świetlną nazywamy to i to
Celem instalacji sygnalizacji na skrzyżowaniu jest to czy tamto.
 jej elementy to to i tamto.

 Systemy jakie się stosuje są takie i siakie.


\section{Szczegóły funkcjonowania wybranych systemów}
\label{s:szczegoly}

\subsection{Podstawowe pojęcia}
\label{ss:pojecia}

\begin{description}
  \item[interwał] -- odcinek czasu, w którym wskazanie danego
  sygnalizatora nie zmienia się
  \item[faza] -- część cyklu sygnalizacji przeznaczona dla grupy
  ruchów, które mogą jednocześnie korzystać ze skrzyżowania; może
  składać się z:
  \begin{itemize}
    \item jednego lub więcej ruchu pojazdów
    \item jednego lub więcej ruchu pieszych
    \item kombinacji pewnej liczby ruchów samochodowych i pieszych
  \end{itemize}
  \item[cykl] -- ustalony ciąg faz

  
\end{description}
\subsection{Sygnalizacja stałoczasowa}
\subsection{Sygnalizacja wzbudzana}

\subsubsection{Detekcja ruchu}
\label{ss:detekcja}

\subsubsection{Opis i parametry systemu}
\label{sec:opis-parametry}
Wykrycie pojazdu przez czujnik powoduje zgłoszenie żądania i włączenie
sygnału zielonego. Sygnał ten jest przedłużany, o ile wykrywane są
następne pojazdy. Jego koniec następuje, jeśli pomiędzy
nadjeżdzającymi pojazdami jest luka o dostatecznie dużej długości
\ang{\mbox{gap-out}} bądź został osiągnięty maksymalny czas sygnału zielonego
dla danego kierunku \ang{\mbox{max-out}}. Druga opcja brana jest pod uwagę
wyłącznie wtedy, gdy wcześniej otrzymano zgłoszenie dla kolidującego
kierunku ruchu.

W celu formalizacji powyższego opisu wprowadzamy następujące definicje:%
\begin{description}
  \item[minimalny zielony] \ang{minimal green} -- najkrótszy
  \item[maksymalny zielony] \ang{maximal green} -- najdłuższy czas,
  przez który dany ruch może otrzymywać zielony sygnał w obecności
  zgłoszenia przeciwnego
  \item[wydłużenie] \ang{passage time} -- czas, o który wydłużany jest
  sygnał zielony po wzbudzeniu czujnika; brak wzbudzenia czujnika
  przez ten czas powoduje wyłączenie sygnału zielonego
\end{description}

\subsubsection{Schemat funkcjonowania podstawowego systemu sygnalizacji wzbudzanej dla pojazdów}
\label{ss:wzbudzana-schemat}

Podstawowy system pełni wzbudzanej sygnalizacji dla pojazdów oparty
jest na następujących założeniach:
\begin{enumerate}
  \item sygnał zielony dla danego ruchu włączany jest wyłącznie w
  wyniku realizacji żądania
  \item czujniki umieszczone są na linii zatrzymania się pojazdów i
  pracują w trybie pulsacyjnym z pamięcią; 
  \item wartości parametrów minimalny i maksymalny zielony oraz
  wydłużenie są stałe
  \item długości interwałów czerwono-żółtego, żółtego i czerwonego
  odstępu są stałe
\end{enumerate}
W celu realizacji powyższych założeń dzielimy interwał zielony na dwie
części \emph{początkową} oraz \emph{rozszerzalną}

\begin{thebibliography}{99}
\addcontentsline{toc}{chapter}{Bibliografia}

% http://ops.fhwa.dot.gov/publications/fhwahop06006/form_dot_1700.htm
\bibitem[FHWA06]{handbook} Robert L. Gordon, Warren Tighe,
\textit{Traffic Control Systems Handbook}, Federal Highway
Administration, 2006.

% http://ops.fhwa.dot.gov/publications/fhwahop08024/doc_page.htm
\bibitem[FHWA08]{timing} Peter Koonce i in., \textit{Traffic Signal
  Timing Manual}, Federal Highway Administration, 2008.



\bibitem[Blar16]{eb1} Elizjusz Blarbarucki, \textit{O pewnych
    aspektach pewnych aspektów}, Astrolog Polski, Zeszyt 16, Warszawa
  1916.
\end{thebibliography}

\end{document}

%%% Local Variables:
%%% mode: latex
%%% TeX-master: t
%%% coding: utf-8
%%% End:
